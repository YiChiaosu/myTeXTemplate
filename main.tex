\documentclass[11pt,b5paper]{book}
\usepackage{xeCJK}
\usepackage{xeCJKfntef}
\usepackage{amsmath}
\usepackage{amsthm} 
\usepackage{amssymb}
\usepackage{multicol}
\usepackage{booktabs}
\usepackage{graphicx}
\usepackage{float}
\usepackage{xcolor}
\usepackage[margin=1.5cm]{geometry}
\usepackage{stackengine}

\setCJKmainfont{cwTeXMing}
\newCJKfontfamily\Kai{cwTeXKai}
\newCJKfontfamily\FangSong{cwTeXFangSong}
\newcommand{\xiaosihao}{\fontsize{12pt}{\baselineskip}\selectfont}
\newcommand{\normal}{\fontsize{11pt}{\baselineskip}\selectfont}
\renewcommand{\baselinestretch}{1.2}
\renewcommand{\CJKglue}{\hskip 1.2pt plus 0.08\baselineskip}

\newcommand{\EC}[1]{\xiaosihao #1 \normal}
\newcommand{\alias}[2]{\stackunder[1pt]{#1}{\tiny #2 \normalsize}}

\title{色即是空}
\author{蘇意喬}
\begin{document}
	\begin{titlepage}
	\centering
	\includegraphics[width=0.15\textwidth]{example-image-1x1}\par\vspace{1cm}
	{\scshape\LARGE Columbidae University \par}
	\vspace{1cm}
	{\scshape\Large Final year project\par}
	\vspace{1.5cm}
	{\huge\bfseries Pigeons love doves\par}
	\vspace{2cm}
	{\Large\itshape John Birdwatch\par}
	\vfill
	supervised by\par
	Dr.~Mark \textsc{Brown}

	\vfill

	% Bottom of the page
	{\large \today\par}
	\end{titlepage}

	\maketitle
	\tableofcontents
	\chapter{This is Nothing}
	\begin{flushright}
		\begin{minipage}[r]{0.7\textwidth}
			\begin{flushright}
			``\textit{Parents of young organic life forms should be warned, that
			towels can be harmful, if swallowed in large quantities.}''
			\\[5pt]
			\rightline{{\rm --- Douglas Adams}}
			「{\FangSong 知之為知之,不知為不知是知也}」
			\\[5pt]
			\rightline{{\rm --- {\Kai 孔子}}}
		\end{flushright}
		\end{minipage}
	\end{flushright}
	
	\section{向量空間}
		\begin{align}
			\label{trivial GCD}
			\gcd (1,2) =\gcd (1,3)=\cdots=\gcd(1,n)
		\end{align}
	\section{第二}
題目:$\forall\ x \in \mathbb{N}$ $F(x)$ 在有限次數會停止。\\
反證法:存在一個$x$會使的$F(x)$無限遞迴,即$F(x)=F(\frac{5x+3}{2})$
解這個$F(x)$若進行k次遞迴
\[(\frac{5}{2})^kx+\frac{3}{2}(\frac{5}{2})^{k-1}+\frac{3}{2}(\frac{5}{2})^{k-2}+\cdots+\frac{3}{2}\]
\[(\frac{5}{2})^k x + \frac{3}{2}\frac{1-(\frac{5}{2})^{k}}{1-\frac{5}{2}} = (\frac{5}{2})^k x - 1 +(\frac{5}{2})^k\]
如果可以做無限次那麼x+1必須有無限個2的倍數故矛盾
\end{document}


